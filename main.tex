\documentclass{article}
\usepackage[utf8]{inputenc}

\title{Elaboration I}
\author{Thomas Kongonis 44618468}
\date{August 2019}

\begin{document}

\maketitle

\tableofcontents


\section{Reassertion of Problem}
\begin{itemize}

\item{\textbf{Nature of problem:} The problem that is being posed throughout this process is specifically a textual
analysis problem.}

\begin{itemize}

\item{\textbf{Macro Process:} The process that is being posed for analysis and elaboration is related to the difference within multiple translations of eastern philosophical texts.}

\item{\textbf{Context:} This concerns specifically east Asian languages for the scope of this proof of concept. The key text that will be utilised as the exemplar of this process will be the Laozi or Dao de Jing as it is also called.}

\end{itemize}

\end{itemize}


\section{Breakdown of Processes}


\begin{itemize}

\item{\textbf{First Step- Finding and Reading of Text:} The Laozi has 81 chapters, for this example chapter one will be utilised after using the internet or library to find a translation.}
\item{\textbf{Second Step-Diagnosing Translation Problem:} The opening sentence of this text was taken from an open source website 'ctext.org/dae-de-jing/'. This translation opens with the quote "\textit{The Dao that can be trodden is not the enduring and unchanging Dao.}"}

\item{\textbf{Third Step-Consulting Secondary Translation} In this stage i would consult a secondary translation, this is because the first translation is a bit clunky and quite confusing. Another permutation of words is necessary for my understanding. for the sake of this example i will utilise the most common translation of this sentence, which is "\textit{The Dao that can be named is not the eternal Dao}."}
\item{\textbf{Fourth Step-Comparison:} In this step, the key points of difference would be diagnosed. In this case we have the words "trodden" and "named" alongside "enduring and unchanging" and "eternal". These appear quite different in meaning even though there is a common thread.}
\item{\textbf{Fifth Step-Draw Conclusions:} In this step, these two words could also be compared with other synonyms out of context of the text for greater understanding. This would normally be done on pen and paper.}
\item{\textbf{Sixth Step- Repeat Until Clarity is Reached:} The issue with this approach is that if i was not happy with my understanding of these translations, then i would have to repeat and continue finding new translations. This is a significant pain point and can become increasingly tedious.}


\end{itemize}

\section{Concept}

\begin{itemize}

\item{\textbf{Converting Pains to Gains:} This entire process listed above is not one that can be totally avoided, however it can be automated and made significantly easier with the existence of textual analysis technology that is able to offer alternate translations upon highlighting a particular word and it consults a database of alternate translations that could include 4 or 5. This way, the synonyms could be assessed within the context of the text itself and meaning would not be lost.}

\end{itemize}

\section{Applicable technology}


\begin{itemize}

\item{\textbf{Textual Analysis Tools}}
\begin{itemize}
\item{There is a number of textual analysis tools that are available that are both proprietary and open source. These include Nvivo, Voyant and even simple word search functions on internet browsers or Pdf viewers.}
\item{An issue with this however, is that many of these tools are either incredibly rudimentary and do not offer the ability to automate processes or, the tools are primarily for text data mining and comparison of words for example frequency searches.}
\item{What this means is that multiple translations can be assessed at the same time however each translated term would need to be discovered and entered firstly. Furthermore, due to the difficulty in translation, synonym searches may not accurately find particular terms in the way a human could.}

\end{itemize}
\item{\textbf{Bibliographic software}}
\begin{itemize}
\item{Textual analysis functions do exist within this software and they are a really good storage system for a multitude of sources. However, they do not really offer a means to make the process more streamlined in the way i am considering.} 
\end{itemize}

\end{itemize}

\end{document}
